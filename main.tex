%
% The PyJournal latex template
% 2020-10-30
% Template Authors: Jan Vrba, Matous Cejnek
%
\documentclass{pjtemplate}


\title{\LaTeX ~template for the PyJournal}
\author{Jan Vrba$^2$, Matous Cejnek$^1$}

% Place 3-7 keywords separated by commas.
\keywords{
    Python, \LaTeX, Journal
}


\affiliation{
\begin{enumerate}
    \item Corresponding author; Dept. of Instrumentation and Control Engineering, Faculty of Mechanical Engineering, Czech Technical University in Prague, Czech Republic,\\ \texttt{matous.cejnek@fs.cvut.cz}
    \item Department of Computing and Control Engineering, University of Chemistry and Technology in Prague, Czechia,\\ \texttt{jan.vrba@vscht.cz}
\end{enumerate}
}

% Use only if your have to provide funding information on the cover page. Leave blank otherwise.
\funding{Supported by grant XXX-XXX-XXX-XXX-XXX-XXX.}

\begin{document}

\maketitle

\begin{abstract}
The abstract should consist from 5-15 sentences describing your paper. Please introduce here following aspects of your paper in the following order: field, purpose, methodology, results, impact. Every aspect should be described by 1-3 sentences. Avoid citing and cross-referencing in the abstract.
\end{abstract}


\section{Introduction}

This is a \LaTeX~template~\cite{lamport1986latex} for the PyJournal -- journal dedicated to the Python \cite{van1995python} programming language.

\section{Sections and subsections}
All sections (except Acknowledgment) should be numbered with Arabic numerals.

\subsection{Subsections}
Subsections should be numbered according to the parent section with Arabic numerals.

\section{Tables, images and equations}
Only this template can be used for the preparation of your manuscript. Please follow all instructions in this template. Equations can be placed inline:\\ 
$c^2~=~a^2~+~b^2$; or in a new line with number for referencing:
\begin{equation}
   c^2 = a^2 + b^2 
\end{equation}
All symbols used in equations have to be properly introduced.


The figures suppose to be placed on relevant place in the body of the paper. See Figure~\ref{fig:example} for example. Keep in mind that the resolution and size of figures should be reasonable.

\begin{figure}[!ht]
\centering
\includegraphics[width=1.\textwidth]{example.png}
\caption{All figures should have relevant caption.}
\label{fig:example}
\end{figure}


The tables should look like the Table~\ref{tab:example}.

\begin{table}[!ht]
\centering
\caption{Example table displaying random data for random fictive persons}
\label{tab:example}

\begin{tabular}{c|c|c|c|c}
      & minimum & maximum & mean & std \\ \hline
Alice & 131     & 175     & 150  & 6   \\
Bob   & 158     & 211     & 201  & 8   \\
Carla & 136     & 350     & 183  & 10 
\end{tabular}
\end{table}




\section{Algorithms and snippets}

All code snippets should be placed via \emph{lstlisting} environment. Use caption and and label as shown in example Listing~\ref{lst:example}.


\begin{lstlisting}[caption={The following code snippet displays a solution of the Fizz Buzz problem.},label={lst:example},language=Python]
for number in range(1,100):
    text = "Fizz" if not number % 3 else ""
    text = text + "Buzz" if not number % 5 else text
    text = text if text else number
    print(text, end=" ")
\end{lstlisting}


All algorithms should be displayed via \emph{algorithm2e} environment. Use caption and and label as shown in example Algorithm~\ref{alg:example}.

\begin{algorithm}[H]
\SetAlgoLined
\KwResult{Write here the result }
 initialization\;
 \While{While condition}{
  instructions\;
  \eIf{condition}{
   instructions1\;
   instructions2\;
   }{
   instructions3\;
  }
 }
 \caption{Algorithm example}\label{alg:example}
\end{algorithm}




\section{Conclusion}

Make sure that you double check all requirements before you submit your manuscript.


\section*{Acknowledgement}

Acknowledgment is a place where you should mention all external support used for creation of the paper. This section should be written in the third person.


\bibliographystyle{unsrt}
\bibliography{references}
\end{document}
